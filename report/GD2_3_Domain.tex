For the purpose of this study, we generated three variations on a Space Battle game. In each of these variations, players are made to control a ship and compete against an AI-controlled opponent. The player is able to control the rotation of the ship, apply thrust, and fire a finite-number of missiles. If the player travels off the side of the screen, they re-appear on the opposite side. A number of pickups are distributed around the map, and any ship that collects the pickup is awarded a number of points. In all of the variations, the player is also able to score points by shooting the opposing ship. The game ends after a set amount of time, with the ship with the most points being classed as the winner.

The first variation on the base game was the inclusion of simple asteroids. These asteroids are distributed randomly at the start of the game, and are destroyed when they are shot or a ship collides with them. Players are given no explicit reward for shooting asteroids, and have their score penalised for colliding with them. The second variation also includes asteroids, but they split rather than get destroyed when they collide with missiles or ships.

The intention of these variants was twofold; firstly, we hoped that the inclusion of asteroids would force the players to move around the screen more, and secondly we hoped that by introducing additional complexity to the game it would be easier to distinguish between skilled and unskilled players.

\begin{figure*}
	\centering
	\caption{Screenshots from the three game modes.}
	\begin{subfigure}[b]{0.3\textwidth}
		\center
		\includegraphics[scale=0.33]{resources/gamemode2}
		\caption{Without asteroids}
	\end{subfigure}
	\begin{subfigure}[b]{0.3\textwidth}
		\center
		\includegraphics[scale=0.33]{resources/gamemode1}
		\caption{With simple asteroids}
	\end{subfigure}
	\begin{subfigure}[b]{0.3\textwidth}
		\center
		\includegraphics[scale=0.33]{resources/gamemode0}
		\caption{With splitting asteroids}
	\end{subfigure}
\end{figure*}
	