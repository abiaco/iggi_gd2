\subsection{Further work}

The researchers acknowledge the limited generalizability of our results due to the small number of participants. Future studies should focus on the more detailed and precise measurement of participants’ game enjoyment which was beyond the scope of this paper.

This study does show that automated AI methods and analytics can be very helpful to assessing, at least at a broad level, how good (or bad) a game mode can be. By immediately discarding, without requiring human input, all the bad variations of a game, designers can focus on tweaking the ones automated testing deem as good. This is far from a completely tapped out research area, as digitizing the human factor is not a solved task.

\subsection{Discussing results}

After human trials more players found the game version with splitting asteroids to be slightly worse than the other two. By looking at the scores collected, we can see that's the only version where players could reliably lose points and be punished for playing the game. This could have led to less enjoyment due to frustration at not only playing against the AI, but also playing against the environment at the same time. The reason behind failing to reject the null hypothesis could be attributed to many factors. The first sample could be the alloted time the participants had to play each game. Four trials for each version may not have been enough to assess or perceive the differences of each version. The second problem could have been with the participants' age. One can see that the mean age (M=38.5) is rather high, being outside of the main demographics of gamers. Thus there is a possibility of the game being too difficult for them. Unfortunately, due to the limited time we have we couldn't fully take advantage of all the available data.

It is possible that the sampled player base is less competitive or less interested in video games as a whole, which is very likely due to the average age and demographic (academic staff and students) of the people tested. This could have skewed results towards liking simpler versions of the game. This would require further research with a more varied selection of people.

By having two almost identical agents, one crippled by less thinking time than the other, games can be tweaked to either favour skilled play for competitive games, or be less reliant on ability for more casual games, where pure random choice leads to viable solutions. This study put emphasis on having skillful game design, rewarding good players and punishing bad players. As mentioned above, different personality and player types will find this polarising. 

This study does show that AI can be used to aid game designers in their search for better games and that more resources should be invested into this area.