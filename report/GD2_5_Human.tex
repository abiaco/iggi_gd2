\subsection{Methodology}

32 participants were recruited through opportunity sampling (21 male and 11 female; 29 right handed, 2 left handed and 1 ambidextrous; ages ranging between 15 to 61 with $\mu= 38.66$ and  $\sigma = \pm 13.55$). All participants provided written consent prior to the experiment and were properly debriefed about the purposes of the experiment afterwards. A questionnaire was created to record the participants' responses. No sensitive information was asked, and the data recorded was fully anonymised. 

Participants played four trials of each version of Space Battle.

After each version of the game was played participants were asked to answer questions regarding that specific version. For counterbalancing we created 6 subconditions (ABC, ACB, BAC, BCA, CAB, CBA) in order to account for possible order-fatigue effects. These procedures were created to reflect best practices in research design.


\subsection{Results}

In order to decide whether parametric or non-parametric tests should be performed, tests of normality were performed. Both the Kolmogorov-Smirnov and Shapiro-Wilk tests suggest that our data significantly deviates from the normal distribution ($p < 0.001$), meaning that non-parametric tests should be used.

Using Friedman's Test were were unable to find any significant results comparing the variations introduced into the game and the player's enjoyment of it ($\chi^2(2) = 3.06$, $p \dom 0.05$).